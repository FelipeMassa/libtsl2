\documentclass[letterpaper]{article}
\usepackage{bargar}
\usepackage{amsmath}
\usepackage{euler}

\student{Joshua Holbrook}
\assignment{Equations for Needle Probe Action}
\duedate{\today}
\coursename{Thermal Systems Lab}
\coursenumber{ME 415}

\begin{document}

\section{The Heat Equation, generalized:}
\begin{equation}
-k\nabla^2T = \rho C \frac{\partial T}{\partial t}
\end{equation}

\section{The exact solution to the heat equation with an infinite line source with
constant heat flux:}
\begin{equation}
T(r, t) = -\frac{q}{4\pi k} \textrm{Ei}\left( -\frac{r^2}{4kt} \right)
\end{equation}
Note that \(q\) is the heat flux per linear distance.

\section{The ``long-time'' solution to the same problem:}
\begin{equation}
T(r,t) = \frac{q}{4\pi k}\ln\left( \frac{4kt}{r^2} - \frac{\gamma q}{4\pi k} \right)
\end{equation}
Note that \(\gamma\) is the ``Euler-Masceroni constant.''

\section{Using this solution to solve for \(k\):}
\begin{equation}
k = \frac{q}{4\pi} \left( \frac{dT}{d\left(\ln( t)\right)} \right)^{-1}
\end{equation}
\(\frac{dT}{d\left(\ln( t)\right)}\) may be found using a linear curve fit.

\section{An Analogous Approach applied to the Cooling Curve:}
\begin{equation}
k = -\frac{q}{4\pi} \left( \frac{dT}{d\left(\ln( t_{\textrm{cool}})\right)} \right)^{-1}
\end{equation}
\(t_{\textrm{cool}} = t - t_{\textrm{heated}}\), where \(t_{\textrm{heated}}\)
is the length of time the sample was heated. In other words, for the cooling
curve, reset \(t_0\).

\section{Calculating \(q\), heat flux per linear distance (W/m) from the
given voltage during the heating curve:}
\begin{equation}
\left. \left(\frac{V^2}{R}\right) \middle/ l \right.
\end{equation}
The ``voltage'' column from the apparatus results is given in millivolts. 
\(R = 0.7902 \Omega\), and \(l = 0.120 m\). \(V\) should be averaged over the
heating curve portion of the experiment.



\end{document}
